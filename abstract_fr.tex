\begin{otherlanguage}{french}
    
Les objets connectés représentent l'une des principales cibles de la cybercriminalité. Les raisons en sont multiples : d'abord, pour des raisons commerciales, les fabricants peuvent vendre des produits vulnérables qui posent des problèmes de sécurité. Deuxièmement, de nombreux appareils IoT sont soumis à des contraintes de performance et ne disposent pas de la puissance nécessaire pour exécuter des logiciels de sécurité. Enfin, l'hétérogénéité des applications, du matériel et des logiciels élargit la surface d'attaque.
Pour parer à ces menaces, l'IoT a besoin de technologies de sécurité et de préservation de la vie privée sur mesure.

 En ce qui concerne la protection de la vie privée, \emph{le contrôle d'usage} donne aux utilisateurs la possibilité de spécifier comment leurs données peuvent être utilisées et par qui. Le contrôle d'usage étend le contrôle d'accès classique en introduisant des \emph{obligations}, c'est-à-dire des actions à effectuer pour obtenir l'accès, et des \emph{conditions} qui sont liées à l'état du système, comme la charge du réseau ou le temps.
Cette thèse vise à apporter des réponses aux défis de l'internet des objets en termes de performance, de sécurité et de respect de la vie privée. Pour cela, les registres distribués (DLT) constituent une solution prometteuse aux contraintes de l'internet des objets, en particulier pour les micro-transactions, notamment par leur caractère décentralisé. Cela se traduit par trois contributions:
1. un ensemble de technologies pour des transactions sans frais préservant la vie privée, conçu pour passer à l'échelle;
2. une méthode d'intégration du contrôle d'usage et des registres distribués pour permettre une protection efficace des données des utilisateurs;
3. un modèle étendu pour le contrôle d'usage dans les systèmes distribués, afin d'y ajouter le contrôle de flux décentralisé et les aspects liés à l'internet des objets.
Une preuve de concept de l'intégration (2) a été mise en place pour démontrer la faisabilité et effectuer des tests de performance. Il s'appuie sur IOTA, un registre distribué qui utilise un graphe orienté acyclique pour son graphe de transactions au lieu d'une \emph{blockchain}. Les résultats des tests de performance sur un réseau privé montrent une diminution d'environ 90\% du temps nécessaire pour effectuer des transactions et pour évaluer des politiques de contrôle d'usage, dans le cas où ce dernier est intégré au réseau.

\end{otherlanguage}