\begin{itemize}
    \item[] \textbf{DApp}: DApps are open-source, decentralized applications that can operate autonomously and without human intervention. DApps make use of cryptocurrencies or tokens, are executed in a network of computers and store outputs in public ledgers \cite{Andoni2018};
    \item[] \textbf{DIFC}: Decentralized information flow control (DIFC) allows users to control the flow of their
    information without imposing the rigid constraints of a traditional
    multilevel security system \cite{Myers1997}; 
    \item[] \textbf{DLT}: Distributed Ledger Technology. A distributed ledger is a register containing a set of transactions.
    This ledger instead of being stored in a single place, the central
    server, is duplicated on a set of machines. The fact that the
    ledger is copied identically multiple times makes it expensive
    to modify it: it requires to be changed on every node of the network. \cite{Quiniou2019};
    \item[] \textbf{DPoS}: Delegated Proof of Stake (DPoS) is a variant of PoS in which "all the stakeholders vote to choose some nodes as witnesses and delegates. Witnesses are responsible and rewarded for creating new blocks. The delegates are responsible for maintaining the network
    and proposing changes such as block sizes, transaction fees, or reward amount" \cite{Salimitari2020};
    \item[] \textbf{Hashgraph}: Hedera Hashgraph is a distributed ledger technology that has a new form of distributed consensus \cite{Baird2018}. It provides fast and ordered transactions, as well as secure infrastructure to run decentralized applications; 
    \item[] \textbf{IFC}: Information Flow Control (IFC) \cite{Denning1976} is 
    "a concept requiring that information
    transfers within a system be controlled so that information in
    certain types of objects cannot, via any channel within the
    system, flow to certain other types of objects" \cite{rfc4949}  ;
    \item[] \textbf{IIoT}: Industrial Internet of Things, refers to the use of IoT technologies in the manufacturing industry \cite{Boyes2018};
    \item[] \textbf{IoT}: Internet of Things, the ever-growing network of physical objects that feature an IP address for internet connectivity, and the communication that occurs between these objects and other Internet-enabled devices and systems \cite{Berte2018}. It is to be noted that the Internet of Things has manifold definitions \cite{Atzori2010}; 
    \item[] \textbf{IOTA}: IOTA is a cryptocurrency for the Internet-of-Things (IoT) industry. The main feature of this
    novel cryptocurrency is the tangle, a directed acyclic graph (DAG) for storing transactions. It offers features that are required to establish a machine-
    to-machine micropayment system \cite{Popov2017}.
    \item[] \textbf{GDPR}: European \emph{General Data Protection Regulation} \cite{EUdataregulations2018};
    \item[] \textbf{Metadata}: Metadata is structured information that describes, explains, locates, or otherwise makes it
    easier to retrieve, use, or manage
    an information resource. Metadata
    is often called data about data or 
    information about information \cite{NISO2004};
    \item[] \textbf{PoA}: The proof of authority (PoA) is a consensus algorithm used most of the
    time for private blockchains. It makes it possible to designate
    nodes of the network as validators, these nodes having the role
    of determining the state of the ledger for the entire network \cite{Quiniou2019}. It significantly increases the speed of
    transaction validations, even if centralization is increased \cite{Quiniou2019};
    \item[] \textbf{PBFT}: In Practical byzantine fault tolerance (PBFT), all the nodes should participate in the voting process in order to add the next block and the consensus
    is reached when more than two-thirds of all nodes agree upon that block. PBFT can tolerate malicious behavior from up to
    one-third of all nodes to perform normally \cite{Salimitari2020};
    \item[] \textbf{PoET}: Proof of elapsed time (PoET) is a consensus method proposed by Intel that works similarly to PoW but with significantly lower
    energy consumption. In this method, miners have to solve a hash problem similar to that of PoW. However, instead of a
    competition between miners to solve the next block, the winning miner is randomly chosen based on a random wait time.
    The winning miner is the one whose timer expires first \cite{Salimitari2020};
    \item[] \textbf{PoS}: the proof of stake (PoS) is an alternative consensus algorithm to the
    proof of work which provides the right to create the next block
    to an active validator on the network that has deposited units
    of the cryptocurrency of this blockchain. The proof of stake is less energy intensive than proof of work \cite{Quiniou2019};
    \item[] \textbf{PoW}: Proof of Work is a computation race taking the form of a cryptographic puzzle, used by the Bitcoin cryptocurrency to elect the node responsible for writing the next transaction on the ledger \cite{Nakamoto2008}; 
    \item[] \textbf{TEE}: a Trusted Execution Environment (TEE) is a tamper-resistant processing environment that runs on a separation
    kernel. It guarantees the authenticity of the executed code, the
    integrity of the runtime states (e.g. CPU registers, memory
    and sensitive I/O), and the confidentiality of its code, data
    and runtime states stored on a persistent memory \cite{Sabt2015};
    \item[] \textbf{UCON} is a model for data access control that allows for dynamic decision-making based on attributes such as context, time, and user behavior \cite{Park2004};
    \item[] \textbf{UCS}: Usage Control System, an entity responsible for monitoring the data usage and dissemination, according to user-defined policies;
    \item[] \textbf{XACML}: The eXtensible Access Control Markup Language (XACML) is an XML-based standard markup language for specifying access control policies. The standard, published by OASIS, defines a declarative fine-grained, attribute-based access control policy language, an architecture, and a processing model describing how to evaluate access requests according to the rules defined in policies \cite{Axiomatics2023}.
\end{itemize}