IoT devices represent one of the major targets for malicious activities. The grounds for this are manifold: first, to reduce the cost of security, manufacturers may sell vulnerable products, leaving users with security concerns. Second, many IoT devices have performance constraints and lack the processing power to execute security software. Third, the heterogeneity of applications, hardware, and software widens the attack surface.
As a result, IoT networks are subject to a variety of cyber threats. To counter such a variety of attacks, the IoT calls for security and privacy-preserving technologies.
%Solutions
 For privacy concerns, \emph{usage control} grants the users the power to specify how their data can be used and by whom. Usage control extends classic access control by introducing \emph{obligations}, i.e., actions to be performed to be granted access, and \emph{conditions} that are related to the system state, such as the network load or the time.
%Main contributions 
This thesis aims to provide answers to the challenges in the Internet of Things in terms of performance, security and privacy. To this end, \emph{distributed ledger technologies} (DLTs) are a promising solution to Internet of Things constraints, in particular for micro-transactions, due to the decentralization they provide. This leads to three related contributions:
1. a framework for zero-fee privacy-preserving transactions in the Internet of Things designed to be scalable;
2. an integration methodology of usage control and distributed ledgers to enable efficient protection of users' data;
3. an extended model for data usage control in distributed systems, to incorporate decentralized information flow control and IoT aspects.
A proof of concept of the integration (2) has been designed to demonstrate feasibility and conduct performance tests. It is based on IOTA, a distributed ledger using a directed acyclic graph for its transaction graph instead of a blockchain. The results of the tests on a private network
show an approximate 90\% decrease in the time needed to push transactions
and make access decisions in the integrated setting.